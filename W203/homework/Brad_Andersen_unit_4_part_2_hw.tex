\documentclass[12pt,a4paper]{article}
\usepackage[inner=1.5cm,outer=1.5cm,top=2.5cm,bottom=2.5cm]{geometry}
\usepackage{graphicx}
\graphicspath{ {./images/} } 
\usepackage[english]{babel}
\usepackage{amsmath}
\usepackage{amssymb}
\numberwithin{equation}{subsection}
\usepackage{hyperref}
\usepackage[utf8]{inputenc}

\def\doubleunderline#1{\underline{\underline{#1}}}

\title{Statistics for Data Science \\
    Unit 4 Part 2 Homework: Continuous Random Variables}
\author{Brad Andersen \\
    W203 Section 4}
\date{February 6, 2019}

\begin{document}

\maketitle

\begin{enumerate}

% ----- Question 1: Processing Pasta ------------------------------------------

\item \textbf{Processing Pasta}

A certain manufacturing process creates pieces of pasta that vary by length.  Suppose that the length of a particular piece, $L$, is a continuous random variable with the following probability density function.

$$f(l) = \begin{cases} 0, &l \leq 0 \\
l/2, &0 < l \leq 2 \\ 
0, &2 < l
\end{cases}
$$

\begin{enumerate}
\item[(a)] \textbf{Write down a complete expression for the cumulative probability function of $L$.}

Identifying the antiderivative of the function for range (0,l]:

\begin{equation*}
\begin{split}
F_{L}(l)dl & = \int_0^lf(m)dm \\
& = \int_0^l\dfrac{m}{2}dm \\
& = \int_0^l\dfrac{m^{2}}{4} \\
\end{split}
\end{equation*}

Calculating for the interval (0,l]:

\begin{equation*}
\begin{split}
F_{L}(l)dl & = \left[\dfrac{m^{2}}{4}\right]_0^l \\
& = \left[\dfrac{l^{2}}{4} - \dfrac{0^{2}}{4}\right] \\
& = \dfrac{l^{2}}{4} \\
\end{split}
\end{equation*}
$$\text{For} \:l \in (0,2]\:F_{L}(l)dl = \dfrac{l^{2}}{4}$$ \\

Applying this equation to complete the cumulative probability function:

\begin{equation*}
F_{L}(l) = \begin{cases} 0, &l \leq 0 \\
\dfrac{l^{2}}{4}, &0 < l \leq 2 \\ 
1, &2 < l
\end{cases}
\end{equation*}

\item[(b)]\textbf{Using the definition of expectation for a continuous random variable, compute the expected length of the pasta, $E(L)$.}

\begin{equation*}
\begin{split}
E(L) &= \int_{-\infty}^{\infty}l \cdot f(l)dl \\
& = \int_0^2 l \cdot \dfrac{l}{2}dl \\
& = \dfrac{1}{2}\int_0^2 l^2 dl \\
& = \dfrac{1}{2}\bigg(\dfrac{l^3}{3}\bigg)\bigg\rvert_{l = 0}^{l = 2} \\
& = \dfrac{1}{2}\bigg(\dfrac{8}{3}\bigg) \\
& = \doubleunderline{\dfrac{4}{3}}
\end{split}
\end{equation*}

\end{enumerate}

% ----- Question 2: The Warranty is Worth It ----------------------------------

\item \textbf{The Warranty is Worth It}

Suppose the life span of a particular (shoddy) server is a continuous random variable, T, with a uniform probability distribution between 0 and 1 year.  The server comes with a contract that guarantees you money if the server lasts less than 1 year.  In particular, if the server lasts $t$ years, the manufacturer will pay you $g(t)= \$100(1-t)^{1/2}$.  Let $X = g(T)$ be the random variable representing the payout from the contract.

\textbf{Compute the expected payout from the contract, $E(X) = E(g(T))$.}

Given that T is a continuous random variable with uniform distribution on the interval [0 years, 1 year], we can represent the probability density function as follows:

$$f(t;0,1) = \begin{cases} \dfrac{1}{1 - 0}, &0 \leq t \leq 1 \\
0, &otherwise
\end{cases}
$$

Validating uniformity, the expected value of T can be calculated as follows:

\begin{equation*}
\begin{split}
E(T) &= \int_{-\infty}^{\infty}t \cdot f(t)dt \\
& = \int_{0}^{1}t \cdot f(t)dt \\
& = \int_{0}^{1}t \cdot 1\cdot dt \\
& = \int_{0}^{1}\dfrac{t^2}{2} \\
\end{split}
\end{equation*}

Calculating for the interval [0, 1]:

\begin{equation*}
\begin{split}
E(T) &= \dfrac{1^2}{2} - \dfrac{0^2}{2} \\
& = \dfrac{1}{2} \\
\end{split}
\end{equation*}

The expected value of T is 0.5 years.

Given that $g(T)$ represents the function of a random variable, it itself is a random variable.  The probability density function of $g(t)$ is as follows:

\begin{equation*}
g(t) = \begin{cases} 0, &t < 0 \\
\$100(1-t)^{1/2}, &0 \leq t \leq 1 \\ 
0, &1 < t
\end{cases}
\end{equation*}

The expected value of $g(t)$ can be calculated as follows:

\begin{equation*}
\begin{split}
E(g(t)) &= \int_{-\infty}^{\infty}g(t) \cdot f(t)dt \\
&= \int_{0}^{1}(\$100(1-t)^{1/2}) \cdot 1 \cdot dt \\
&= \$100 \int_{0}^{1}(1-t)^{1/2} dt \\
&= \$100 \int_{0}^{1}-\dfrac{2}{3}(1-t)^{3/2} \\
&= \$100 \int_{0}^{1}-\dfrac{2 (1-t)^{3/2}}{3} \\
&= \$100 \bigg(-\dfrac{2(1-1)^{3/2}}{3}\bigg) - \$100 \bigg(-\dfrac{2(1-0)^{3/2}}{3}\bigg) \\
&= \$100 \bigg(0\bigg) - \$100 \bigg(-\dfrac{2}{3}\bigg) \\
&= \dfrac{\$200}{3} \\
\end{split}
\end{equation*}

The expected payout from the contract is \$66.67 rounded to the nearest cent.

% ----- Question 3: (Lecture)#Fail --------------------------------------------

\item \textbf{(Lecture)\#Fail}

Suppose the length of Paul Laskowski's lecture in minutes is a continuous random variable C, with pmf $f(t) = e^{-t}$ for $t > 0$.  This is an example of an exponential random variable, and it has some special properties.  For example, suppose you have already sat through t minutes of the lecture, and are interested in whether the lecture is about to end immediately.  In statistics, this can be represented by something called the \textit{hazard rate}:

$$h(t) = \frac{f(t)}{1-F(t)}$$

To understand the hazard rate, think of the numerator as the probability the lecture ends between time $t$ and time $t+ dt$.  The denominator is just the probability the lecture does not end before time $t$.  So you can think of the fraction as the conditional probability that the lecture ends between $t$ and $t + dt$ given that it did not end before $t$.

\textbf{Compute the hazard rate for C.}

The probability density function for $f(t)$ can be represented as follows:

$$f(t) = \begin{cases} e^{-t}, &0 < t \\
0, &otherwise
\end{cases}
$$

Computing $F(t)$:

\begin{equation*}
\begin{split}
F(t) &= \int_{-\infty}^{\infty}f(t)dt \\
&= \int_{-\infty}^{\infty}(e^{-t})dt \\
&= \int_{0}^{\infty}-e^{-t} \: \text{where} \: t > 0 \\
\end{split}
\end{equation*}

Substituting values, the hazard rate for C can be calculated as follows:

\begin{equation*}
\begin{split}
h(t) &= \dfrac{e^{-t}}{1 - (-e^{-t})} \: \text{where} \: t > 0 \\
&= \dfrac{e^{-t}}{1 + e^{-t}} \: \text{where} \: t > 0 \\
\end{split}
\end{equation*}

\end{enumerate}

\end{document}
