\documentclass[12pt,a4paper]{article}
\usepackage[inner=1.5cm,outer=1.5cm,top=2.5cm,bottom=2.5cm]{geometry}
\usepackage{graphicx}
\graphicspath{ {./images/} }
\usepackage[english]{babel}
\usepackage{amsmath}
\usepackage{amssymb}
\numberwithin{equation}{subsection}
\usepackage{hyperref}
\usepackage[utf8]{inputenc}

\def\doubleunderline#1{\underline{\underline{#1}}}

\title{Statistics for Data Science \\
    Unit 4 Part 1 Homework: Discrete Random Variables}
\author{Brad Andersen \\
    W203 Section 4}
\date{January 30, 2019}

\begin{document}

\maketitle

\begin{enumerate}

% ----- Question 1: Best Game in the Casino --------------------------------------------------------

\item \textbf{Best Game in the Casino}

You flip a fair coin 3 times, and get a different amount of money depending on how many heads you get. For 0 heads, you get \$0. For 1 head, you get \$2. For 2 heads, you get \$4. Your expected winnings from the game are \$6. 

\begin{enumerate}
\item How much do you get paid if the coin comes up heads 3 times?
\end{enumerate}
Because:
\begin{itemize}
    \item Our coin-flipping activities are divided into three separate \textit{trials}
    \item Each trial flip has only two outcomes: heads or tails
    \item Each trial flip is independent of the others
    \item The probability of heads or tails is consistent from one trial flip to the next
\end{itemize}
our coin-flipping activities can be considered a \textbf{binomial experiment}.\\ \\
Payout is based upon the number of heads resulting in the experiment: 0, 1, 2 or 3.  Probabilities can be calculated using the binomial distribution probability theorem, as follows:
\begin{equation*}
    b(x; n, p) =
    \begin{cases}
    \binom{n}{x}p^{x}(1 - p)^{n - x} & x = 0, 1, 2, \dots, n\\
    0 & \text{otherwise}
    \end{cases}
\end{equation*}
where \textit{x} represents the number of heads results, \textit{n} represents the number of trials in the experiment, and \textit{p} represents the probability of the outcome. \\ \\
For example, calculating the probability that the experiment's results include no heads,
\begin{equation*}
\begin{split}
P(X = 0) = b(0; 3, 0.5) & = \binom{3}{0}p^{0}(1 - 0.5)^{3 - 0} \\
& = \frac{3!}{0!(3 - 0)!} \times 0.5^{0} \times 0.5^{3} \\
& = \frac{6}{6} \times 1 \times 0.125 \\
& = 1 \times 1 \times 0.125 \\
& = 0.125
\end{split}
\end{equation*}
Therefore, the probability mass function for the experiment is as follows: \\ \\
\begin{tabular}{c|cccc}
\textit{x} & 0 & 1 & 2 & 3 \\
\hline
\textit{p}(\textit{x}) & 0.125 & 0.375 & 0.375 & 0.125 \\
\end{tabular} \\ \\
Payout from the results of the experiment can be considered a function of the discrete random variable \textit{X}, itself the results of the coin-flipping experiment.  If \textit{h}(\textit{X}) is this function, it too is a random variable such that \textit{Y} = \textit{h}(\textit{X}).  The derived probability mass function for \textit{Y} is as follows: \\ \\
\begin{tabular}{c|cccc}
\textit{y} & \$0 & \$2 & \$4 & \$\textit{m} \\
\hline
\textit{p}(\textit{y}) & 0.125 & 0.375 & 0.375 & 0.125 \\
\end{tabular} \\ \\
where \textit{m} represents the winnings for experiment results of three heads, the winnings to be calculated.  Given that we know the expected winnings from the experiment are \$6, winnings for results of three heads can be calculated as follows:
\begin{equation*}
\begin{split}
E(Y) & = E[h(X)] = \sum_{\substack{D}}h(x) \times p(x) \\
& = \$6 = (\$0)(0.125) + (\$2)(0.375) + (\$4)(0.375) + (\textit{m})(0.125) \\
& = \$6 = \$0 + \$0.75 + \$1.50 + (\textit{m})(0.125) \\
& = \$3.75 = (\textit{m})(0.125) \\
& = \$30 = \textit{m}
\end{split}
\end{equation*}
Therefore, \doubleunderline{winnings for experiment results of three heads are \$30}.

\begin{enumerate}
    \item[(b)]Write down a complete expression for the cumulative probability function for your winnings from the game.
\end{enumerate}
\doubleunderline{\begin{equation*}
    \textit{F}(\textit{y}) =
    \begin{cases}
    0.125 & 0 \le y < 1 \\
    0.500 & 1 \le y < 2 \\
    0.875 & 2 \le y < 3 \\
    1 & 3 \le y
    \end{cases}
\end{equation*}}

\end{enumerate}
\end{document}
